\documentclass[10pt]{article}

\usepackage{amsmath,amssymb,amsfonts,amsbsy}
\usepackage{epsfig,array}


\textwidth 17cm \textheight 24cm \setlength{\oddsidemargin}{-3mm}
\setlength{\evensidemargin}{-3mm}
\setlength{\headheight}{-1\baselineskip}
\setlength{\headsep}{-1\baselineskip}

\renewcommand{\section}{\subsection}


\pagestyle{empty}


% Definitions
%================================================================
\def\curl{{\rm curl}\,}
\def\div{{\rm div}\,}
\def\la{\lambda}
\def\ba{{\bf a}}
\def\bb{{\bf b}}
\def\be{{\bf e}}
\def\bj{{\bf j}}
\def\bn{{\bf n}}
\def\bV{{\bf V}}
\def\bJ{{\bf J}}
\def\bv{{\bf v}}
\def\bu{{\bf u}}
\def\bw{{\bf w}}
\def\bH{{\bf H}}
\def\vf{{\bf f}}
\def\bh{{\bf h}}
\def\bU{{\bf U}}
\def\bV{{\bf V}}
\def\bx{{\bf x}}
\def\by{{\bf y}}
\def\bz{{\bf z}}
\def\bX{{\bf X}}
\def\bF{{\bf F}}
\def\bC{{\bf C}}

\def\vx {{\bf x}}
\def\vy {{\bf y}}
\def\vz {{\bf z}}
\def\vv {{\bf v}}
\def\vu {{\bf u}}
\def\vb {{\bf b}}
\def\vc {{\bf c}}
\def\vr {{\bf r}}

\def\pr{{\partial}}
\def\veta{\boldsymbol{\eta}}
\def\bPhi{\boldsymbol{\Phi}}
\def\bxi{\boldsymbol{\xi}}
\def\bnu{\boldsymbol{\nu}}
\def\Bom{\boldsymbol{\Omega}}
\def\pd#1#2{\frac{\displaystyle\partial#1}{\displaystyle\partial#2}}
\def\bfr#1#2{\frac{\displaystyle #1}{\displaystyle #2}}
\def\vec#1{\boldsymbol{#1}}
\def\Bbb{\mathbb}
\def \shalf{{\textstyle \frac{1}{2}}}
\def \half{\frac{1}{2}}
\def \ssum{{\textstyle \sum}}

%=======================================================================



\begin{document}


\begin{center}
{{\bf Numerical Methods for PDEs (Spring 2017)}}
\end{center}


\begin{center}
{\large{\bf Solutions 3}}
\end{center}


\centerline{}




%%%%%%%%%%%%%%%%%%%%%%%%%%%%%%%%%%%%%%%%%%%%% Q1 %%%%%%%%%%%%%%%%%%%%%%%%%%%%%%%%%%%%%%%%%%%%%%%%%%%%%%%%%%%


\noindent

\vskip 0.5cm
\noindent
{\bf Problem 9.} By  expanding $g(x\pm
h)$, $Q(x\pm h)$ in Taylor's series at $x$, show that
\[
\frac{d }{d x}\left(Q(x) \frac{d g}{\pr x}\right)=\frac{1}{h^2}
\left(Q_{+}\left[g(x+h)-g(x)\right]-
Q_{-}\left[g(x)-g(x-h)\right]\right) +O(h^2),
\]
where
\[
Q_{\pm}=\frac{1}{2}\left[Q(x)+Q(x\pm h)\right].
\]


\vskip 0.5cm \noindent
{\bf Solution.} We have
\[
g(x\pm h)=g(x)\pm h g^{\prime}(x) +\frac{h^2}{2} g^{\prime\prime}(x)
\pm\frac{h^3}{6} g^{\prime\prime\prime}(x) +O(h^4).
\]
Hence,
\begin{eqnarray}
&&g(x+h)-g(x)=h g^{\prime}(x)+\frac{h^2}{2} g^{\prime\prime}(x)
+\frac{h^3}{6} g^{\prime\prime\prime}(x) +O(h^4),  \nonumber \\
&&g(x)-g(x-h)=h g^{\prime}(x)-\frac{h^2}{2} g^{\prime\prime}(x)
+\frac{h^3}{6} g^{\prime\prime\prime}(x) +O(h^4),  \label{sq1}
\end{eqnarray}
Also, since
\[
Q(x\pm h)=Q(x)\pm h Q^{\prime}(x) +\frac{h^2}{2} Q^{\prime\prime}(x)
\pm\frac{h^3}{6} Q^{\prime\prime\prime}(x) +O(h^4),
\]
we obtain
\begin{eqnarray}
&&Q_{+}=Q(x)+ \frac{h}{2} Q^{\prime}(x) +\frac{h^2}{4} Q^{\prime\prime}(x)
+\frac{h^3}{12} Q^{\prime\prime\prime}(x) +O(h^4), \nonumber \\
&&Q_{-}=Q(x)- \frac{h}{2} Q^{\prime}(x) +\frac{h^2}{4} Q^{\prime\prime}(x)
-\frac{h^3}{12} Q^{\prime\prime\prime}(x) +O(h^4). \label{sq2}
\end{eqnarray}
Let
\begin{equation}
X=\frac{d }{d x}\left(Q(x) \frac{d g}{\pr x}\right)-\frac{1}{h^2}
\left(Q_{+}\left[g(x+h)-g(x)\right]-
Q_{-}\left[g(x)-g(x-h)\right]\right). \label{sq3}
\end{equation}
We need to show that $X=O(h^2)$. Substitution of (\ref{sq1}) and (\ref{sq2}) into (\ref{sq3}) yields
\begin{eqnarray}
X &=&\frac{d }{d x}\left(Q(x) \frac{d g}{\pr x}\right) \nonumber \\
&&-\frac{1}{h^2}
\Biggl\{
\left(Q(x)+ \frac{h}{2} Q^{\prime}(x) +\frac{h^2}{4} Q^{\prime\prime}(x)
+\frac{h^3}{12} Q^{\prime\prime\prime}(x) +O(h^4)\right)
\!\left(h g^{\prime}(x)+\frac{h^2}{2} g^{\prime\prime}(x)
+\frac{h^3}{6} g^{\prime\prime\prime}(x) +O(h^4)\right)  \nonumber \\
&&-
\left(Q(x)- \frac{h}{2} Q^{\prime}(x) +\frac{h^2}{4} Q^{\prime\prime}(x)
-\frac{h^3}{12} Q^{\prime\prime\prime}(x) +O(h^4)\right)
\!\left(h g^{\prime}(x)-\frac{h^2}{2} g^{\prime\prime}(x)
+\frac{h^3}{6} g^{\prime\prime\prime}(x) +O(h^4)\right)\Biggr\} \nonumber \\
&=&\frac{d }{d x}\left(Q(x) \frac{d g}{\pr x}\right) -\frac{1}{h^2}
\Biggl\{h Q g^{\prime}+\frac{h^2}{2}\left(Q g^{\prime\prime}+Q^{\prime}g^{\prime}\right)+
h^3\left(\frac{1}{6}Q g^{\prime\prime\prime}+\frac{1}{4}Q^{\prime}g^{\prime\prime}
+\frac{1}{4}Q^{\prime\prime}g^{\prime}\right) \nonumber \\
&&-\left(h Q g^{\prime}-\frac{h^2}{2}\left(Q g^{\prime\prime}+Q^{\prime}g^{\prime}\right)+
h^3\left(\frac{1}{6}Q g^{\prime\prime\prime}+\frac{1}{4}Q^{\prime}g^{\prime\prime}
+\frac{1}{4}Q^{\prime\prime}g^{\prime}\right) \right)+ O(h^4) \Biggr\}  \nonumber \\
&=&\frac{d }{d x}\left(Q(x) \frac{d g}{\pr x}\right)-\left(Q g^{\prime\prime}+Q^{\prime}g^{\prime}\right)
+ O(h^2)= O(h^2).
\end{eqnarray}


%%%%%%%%%%%%%%%%%%%%%%%%%%%%%%%%%%%%%%%%%%%%% Q2 %%%%%%%%%%%%%%%%%%%%%%%%%%%%%%%%%%%%%%%%%%%%%%%%%%%%%%%%%%%

\vskip 0.5cm \noindent
{\bf Problem 10.} Consider the two-dimensional heat equation
\begin{equation}
\frac{\partial u}{\partial t}-K \left(\frac{\partial^{2}u}{\partial
x^{2}}+\frac{\partial^{2}u}{\partial
y^{2}}\right)=0 \quad \hbox{for} \quad 0<x<L_{1}, \ \ y<x<L_{2}, \ \ t>0, \label{1}
\end{equation}
subject to the boundary conditions
\[
u(0,y,t)=0, \quad u(L_{1},y,t)=0, \quad
u(x,0,t)=0, \quad u(x,L_{2},t)=0,
\]
and the initial condition
\[
u(x,y,0)=u_{0}(x,y).
\]
At interior grid points $(x_{x},y_{j},t_{n})$, equation (\ref{1}) is approximated by the finite-difference scheme
\[
\frac{w_{kj}^{n}-w_{kj}^{n-1}}{\tau} -K\left(\frac{\delta_{x}^2}{h_{1}^2}
+\frac{\delta_{y}^2}{h_{2}^2}\right)w_{kj}^{n}=0,
\]
where $w_{kj}^{n}$ are approximations to $u(x_{x},y_{j},t_{n})$; $x_{k}=k h_{1}$ for $k=0,1,\dots N_{1}$,
$h_{1}=L_{1}/N_{1}$; $y_{j}=j h_{2}$ for $j=0,1,\dots N_{2}$,
$h_{2}=L_{2}/N_{2}$; $t_{n}=n\tau$ for $n=0,1,\dots$ and $\tau$ in the length of the time step.

\vskip 0.3cm
\noindent
Investigate the stability of this scheme by the Fourier method.

\vskip 0.5cm \noindent
{\bf Solution.}
The perturbation  $z_{kj}$ satisfies the difference equation
\begin{equation}
\frac{z_{kj}^{n}-z_{kj}^{n-1}}{\tau} -K\left(\frac{\delta_{x}^2}{h_{1}^2}
+\frac{\delta_{y}^2}{h_{2}^2}\right)z_{kj}^{n}=0. \label{f7}
\end{equation}
We seek a particular solution of
(\ref{f7}) in the form
\begin{equation}
z^{n}_{k,j}=\rho^{n}e^{iqx_{k}+ipy_{j}}. \label{f8}
\end{equation}
for $q,p\in{\mathbb R}$ and $n=0,1,\dots$
The finite-difference method is stable if
\[
\vert\rho\vert\leq 1 \quad {\rm for \ all } \quad q,p\in{\mathbb R}.
\]
Substituting (\ref{f8}) in (\ref{f7}), we obtain
\[
e^{iqx_{k}+ipy_{j}}\left(\rho^{n}-\rho^{n-1}\right) -
\gamma_{1}\rho^{n}e^{ipy_{j}}
\left(e^{iqx_{k+1}}-2e^{iqx_{k}}+e^{iqx_{k-1}}\right)-
\gamma_{2}\rho^{n}e^{iqx_{k}}
\left(e^{ipy_{j+1}}-2e^{ipy_{j}}+e^{ipy_{j-1}}\right)=0
\]
or, equivalently,
\[
1-\frac{1}{\rho}-\gamma_{1}
\left(e^{iqh_{1}}-2+e^{-iqh_{1}}\right)-
\gamma_{2}\left(e^{iph_{2}}-2+e^{-iph_{2}}\right)=0,
\]
where
\[
\gamma_{1}=\frac{K\tau}{h_{1}^2}, \quad
\gamma_{2}=\frac{K\tau}{h_{2}^2}.
\]
Since
\[
e^{iqh_{1}}-2+e^{-iqh_{1}}=-4\sin^{2} \frac{qh_{1}}{2}, \quad
e^{iph_{2}}-2+e^{-iph_{2}}=-4\sin^{2} \frac{ph_{2}}{2},
\]
we obtain
\[
\rho=\frac{1}{1+4\gamma_{1}\sin^{2} \frac{qh_{1}}{2}
+4\gamma_{2}\sin^{2} \frac{ph_{2}}{2}}.
\]
Evidently, $0<\rho\leq 1$ for all $q$ and $p$. Therefore,
the scheme is unconditionally stable.



%%%%%%%%%%%%%%%%%%%%%%%%%%%%%%%%%%%%%%%%%%%%% Q2 %%%%%%%%%%%%%%%%%%%%%%%%%%%%%%%%%%%%%%%%%%%%%%%%%%%%%%%%%%%

\vskip 0.5cm \noindent
{\bf Problem 11.}
The nonlinear heat equation
\[
u_t -K u_{xx}=f(u) \quad \textrm{for} \quad 0 < x < L, \ \ 0 < t < T
\]
(where $K$ is a constant and $f(u)$ is a given function), subject to the initial and boundary conditions
\[
u(x,0)=u_0(x) \ \ \hbox{for} \ 0 < x < L, \quad u(0,t)=u(L,t)=0  \ \ \hbox{for} \ 0 < t < T,
\]
is solved using the finite-difference method:
\begin{eqnarray}
&&\frac{w_{k,j}-w_{k,j-1}}{\tau}-
K \, \frac{w_{k+1,j}-2w_{k,j}+w_{k-1,j}}{h^{2}}=f(w_{kj}) \ \ \hbox{for} \ \ k=1,2,\dots,N-1 \  \hbox{and} \ j=1,2,\dots,M; \label{pp11} \\
&&w_{k,0}=u_0(x_{k}) \ \ \hbox{for} \ k=0, \dots,N \ \ \hbox{and} \ \ w_{0,j}=w_{N,j}=0 \ \ \hbox{for} \ j=1, \dots,M . \label{pp22}
\end{eqnarray}
Obtain the computation formulae for solving the nonlinear equations (\ref{pp11}) by the Newton method and implement the solution in R.





\vskip 0.5cm \noindent
{\bf Solution.} It is convenient to rewrite Eq. (\ref{pp11}) in the form
\begin{equation}
-\gamma w_{k-1,j}+(1+2\gamma)w_{k,j}-\gamma w_{k+1,j} - \tau f(w_{kj}) - w_{k,j-1}=0 \label{pp33}
\end{equation}
where $\gamma=K\tau/h^2$. In vector form, this can be written as
\begin{equation}
\bPhi(\bw_j)\equiv A\bw_j - \tau \bF(\bw_{j}) - \bw_{j-1}={\bf 0} \label{pp44}
\end{equation}
where
\[
{\bf w}_{j}=\left[
\begin{array}{c}
w_{1,j} \\
w_{2,j} \\
\vdots \\
\vdots \\
\vdots \\
w_{N-1,j}
\end{array}\right], \quad
{\bf F}(\bw_{j})=
\left[
\begin{array}{c}
f(w_{1,j}) \\
f(w_{2,j}) \\
\vdots \\
\vdots \\
\vdots \\
f(w_{N-1,j})
\end{array}\right], \quad
A=\left[
\begin{array}{cccccc}
1+2\gamma & -\gamma &0      &\dots  &\dots &0 \\
-\gamma &1+2\gamma &-\gamma &\ddots  &     &\vdots \\
0      &-\gamma &1+2\gamma &-\gamma &\ddots &\vdots \\
\vdots &\ddots &\ddots &\ddots &\ddots &0 \\
\vdots &       &\ddots &\ddots &\ddots &-\gamma \\
0      &\dots  &\dots  &0   &-\gamma   &1+2\gamma
\end{array}\right].
\]
Equation (\ref{pp33}) is equivalent to
\begin{equation}
\Phi_{i}(\bw_j)\equiv \sum_{m=1}^{N-1}a_{im}w_{m,j}-\tau f(w_{i,j})-w_{i,j-1}=0, \quad i=1,\dots,N-1 \label{pp44}
\end{equation}
where $a_{im}$ are the entries of matrix $A$. Hence,
\begin{equation}
J_{ik}(\bw_j)=\frac{\pr \Phi_{i}}{\pr w_{k,j}} = a_{ik}-\tau f'(w_{k,j})\delta_{ik} \label{pp55}
\end{equation}
where the $J_{ik}$ are the entries of the Jacobian matrix $J$ (see the lecture notes) and $\delta_{ik}$ is the Kronecker delta
($\delta_{ik}=1$ for $i=k$ and $\delta_{ik}=0$ for $i\neq k$). In Newton's method,
we compute a sequence $\{\bw_j^{(s)}\}$ ($s=0,1,\dots$) using the formula
\begin{equation}
\bw_j^{(s)}=\bw_j^{(s-1)}+{\bf r}^{(s)},\label{pp66}
\end{equation}
where ${\bf r}^{(s)}$ is the solution of the linear system
\begin{equation}
J\left({\bw_j}^{(s-1)}\right){\bf r}^{(s)}=-{\boldsymbol{\Phi}}\left({\bw_j}^{(s-1)}\right) \label{pp77}
\end{equation}
The initial approximation $\bw_j^{0}$ is chosen as
\[
\bw_j^{0}=\bw_{j-1}.
\]
Equivalently, equation (\ref{pp77}) can be written
 in the form (cf. Eq. (2.113) in the lecture notes):
\begin{eqnarray}
&&-\gamma r_{k-1,j}^{(s)}
-\gamma r_{k+1,j}^{(s)}+\left(1+2\gamma - \tau f'\left(w_{kj}^{(s-1)}\right)\right)r_{k,j}^{(s)}=  \nonumber \\
&&=-(1+2\gamma)w_{kj}^{(s-1)}+\gamma\left(w_{k+1,j}^{(s-1)}+w_{k-1,j}^{(s-1)}\right)+\tau f\left(w_{kj}^{(s-1)}\right)+w_{k,j-1}  \label{pp88}
\end{eqnarray}
This can be solved using the double sweep method.

\end{document}
